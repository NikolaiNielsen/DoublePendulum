% Nikolai Nielsens "Fysiske Fag" preamble
\documentclass[a4paper,11pt]{article}
\usepackage[english]{babel}
\usepackage[dvipsnames]{xcolor}
\usepackage[margin=0.75in]{geometry}
\usepackage{wrapfig}
\usepackage{Nikolai}
\usepackage{pdfpages}

\usepackage{listings}
\usepackage{color}
\definecolor{Maroon}{RGB}{175,50,53}
\definecolor{OliveGreen}{RGB}{60,128,49}
\definecolor{Orange}{RGB}{245,129,55}

\lstset{ %
	language=python,
	numbers=left,
	stepnumber=1,
	breaklines=true,
	keepspaces=true,
	showstringspaces=false, 
	tabsize=4,
	basicstyle=\footnotesize\ttfamily,
	keywordstyle=\bfseries\color{Maroon},
	commentstyle=\itshape\color{OliveGreen},
	identifierstyle=\color{blue},
	stringstyle=\color{Orange},
}

% Til nummerering af ligninger. Så der står (afsnit.ligning) og ikke bare (ligning)

\newcommand{\coef}[1]{_{[#1]}}

% Header
%\usepackage{fancyhdr}
%\head{}
%\pagestyle{fancy}

%Titel

\begin{document}
	\section{Lagrangian}
	The Lagrangian for the system is given by
	\begin{equation}\label{key}
		L = T - V
	\end{equation}
	with $ T $ being the kinetic energy and $ V $ being the potential energy. These can be further split up into the kinetic and potential energy for each pendulum. We get
	\begin{equation}\label{key}
		T_1 = \frac{1}{2} m_1 (\dot{x}_1^2 + \dot{y}_1^2), \quad T_2 = \frac{1}{2} m_2 (\dot{x}_2^2 + \dot{y}_2^2)
	\end{equation}
	And
	\begin{equation}\label{key}
		V_1 = m_1 g y_1, \quad V_2 = m_2 g y_2
	\end{equation}
	We further choose to use the angles the pendulums make with the negative $ y $-direction as the generalized coordinates. The conversions are
	\begin{equation}\label{key}
		x_1 = r_1 \sin \theta_1, \quad y_1 = -r_1 \cos \theta_1, \quad x_2 = r_1 \sin \theta_1 + r_2 \sin \theta_2, \quad y_2 = -r_1 \cos \theta_1 - r_2 \cos \theta_2
	\end{equation}
	and
	\begin{equation}\label{key}
		\theta_1 = \arctan(y_1/x_1) + \pi/2\ \text{mod}\ 2\pi\quad  \theta_2 = \arctan\frac{y_2-y_1}{x_2-x_1} + \pi/2 \ \text{mod}\ 2\pi 
	\end{equation}
\end{document}