% Nikolai Nielsens "Fysiske Fag" preamble
\documentclass[a4paper,11pt]{article}
\usepackage[english]{babel}
\usepackage[dvipsnames]{xcolor}
\usepackage[margin=0.75in]{geometry}
\usepackage{wrapfig}
\usepackage{Nikolai}
\usepackage{pdfpages}

\usepackage{listings}
\usepackage{color}
\definecolor{Maroon}{RGB}{175,50,53}
\definecolor{OliveGreen}{RGB}{60,128,49}
\definecolor{Orange}{RGB}{245,129,55}

\lstset{ %
	language=python,
	numbers=left,
	stepnumber=1,
	breaklines=true,
	keepspaces=true,
	showstringspaces=false, 
	tabsize=4,
	basicstyle=\footnotesize\ttfamily,
	keywordstyle=\bfseries\color{Maroon},
	commentstyle=\itshape\color{OliveGreen},
	identifierstyle=\color{blue},
	stringstyle=\color{Orange},
}

% Til nummerering af ligninger. Så der står (afsnit.ligning) og ikke bare (ligning)

\newcommand{\coef}[1]{_{[#1]}}

% Header
%\usepackage{fancyhdr}
%\head{}
%\pagestyle{fancy}

%Titel

\begin{document}
	\section{Lagrangian}
	The Lagrangian for the system is given by
	\begin{equation}\label{key}
		L = T - V
	\end{equation}
	with $ T $ being the kinetic energy and $ V $ being the potential energy. These can be further split up into the kinetic and potential energy for each pendulum. We get
	\begin{equation}\label{key}
		T_1 = \frac{1}{2} m_1 (\dot{x}_1^2 + \dot{y}_1^2), \quad T_2 = \frac{1}{2} m_2 (\dot{x}_2^2 + \dot{y}_2^2)
	\end{equation}
	And
	\begin{equation}\label{key}
		V_1 = m_1 g y_1, \quad V_2 = m_2 g y_2
	\end{equation}
	We further choose to use the angles the pendulums make with the negative $ y $-direction as the generalized coordinates. The conversions are
	\begin{equation}\label{key}
		x_1 = r_1 \sin \theta_1, \quad y_1 = -r_1 \cos \theta_1, \quad x_2 = r_1 \sin \theta_1 + r_2 \sin \theta_2, \quad y_2 = -r_1 \cos \theta_1 - r_2 \cos \theta_2
	\end{equation}
	and
	\begin{equation}\label{key}
		\theta_1 = \arctan(y_1/x_1) + \pi/2\ \text{mod}\ 2\pi\quad  \theta_2 = \arctan\frac{y_2-y_1}{x_2-x_1} + \pi/2 \ \text{mod}\ 2\pi.
	\end{equation}
	In the new coordinates, the Lagrangian becomes
	\begin{equation}\label{key}
		L = \frac{m_1+m_2}{2}(\dot{\theta}_1 r_1)^2 + \frac{m_2}{2} (\dot{\theta}_2 r_2)^2 + m_2 r_1r_2\dot{\theta}_1\dot{\theta}_2 \cos(\theta_1-\theta_2) + gr_1(m_1+m_2) \cos\theta_1 + m_2gr_2 \cos\theta_2
	\end{equation}
	Giving us the equations of motion:
	\begin{align}
		\ddot{\theta}_1 &= -\frac{m_2 r_2}{(m_1+m_2)r_1} \bb{\ddot{\theta}_2 \cos(\theta_1-\theta_2) + \dot{\theta}_2^2 \sin(\theta_1-\theta_2)} - \frac{g}{r_1} \sin\theta_1\\
		\ddot{\theta}_2 &= -\frac{r_1}{r_2} \ddot{\theta}_1 \cos(\theta_1-\theta_2) + \frac{r_1}{r_2} \dot{\theta}_1^2 \sin(\theta_1-\theta_2) - \frac{g}{r_2} \sin\theta_2. 
	\end{align}
	or, decoupling them:
	\begin{align*}
		\ddot{\theta}_1 = f(\theta_1, \theta_2, \dot{\theta}_1, \dot{\theta}_2) &= \frac{m_2 r_1 \dot{\theta}_1^2 \sin(\theta_1-\theta_2)\cos(\theta_1-\theta_2) + m_2g\sin\theta_2 \cos(\theta_1-\theta_2)}{(m_1+m_2)r_1 - m_2r_1\cos^2(\theta_1-\theta_2)} \\
		&+  \frac{m_2 r_2 \dot{\theta}_2^2 \sin(\theta_1-\theta_2) - (m_1+m_2)g\sin\theta_1}{(m_1+m_2)r_1 - m_2r_1\cos^2(\theta_1-\theta_2)}\\
		\ddot{\theta}_2 = g(\theta_1, \theta_2, \dot{\theta}_1, \dot{\theta}_2) &= \frac{-m_2r_2\dot{\theta}_2^2 \sin(\theta_1-\theta_2)\cos(\theta_1-\theta_2) + (m_1+m_2) g \sin\theta_1 \cos(\theta_1-\theta_2)}{\frac{r_2}{r_1} \bb{(m_1+m_2)r_1 - m_2r_1\cos^2(\theta_1-\theta_2)}} \\
		&+\frac{- (m_1+m_2) r_1 \dot{\theta}_1^2 \sin(\theta_1-\theta_2) - (m_1+m_2) g \sin\theta_2}{\frac{r_2}{r_1} \bb{(m_1+m_2)r_1 - m_2r_1\cos^2(\theta_1-\theta_2)}}
	\end{align*}
	Next we use the fourth order Runge-Kutta method for advancing in time. Let $ \dot{y}=f(y, t) $, then
	\begin{align}
		y(t + \Delta t) &= y(t) + \frac{1}{6} (k_1 + 2k_2 + 2k_3 + k_4), \\
		k_1 &= \Delta t \ f(y(t), t),\\
		k_2 &= \Delta t \ f(y(t) + k_1/2, t+\Delta t/2), \\
		k_3 &= \Delta t \ f(y(t) + k_2/2, t+\Delta t/2), \\
		k_4 &= \Delta t \ f(y(t) + k_3, t+\Delta t).
	\end{align}
\end{document}